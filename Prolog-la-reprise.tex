\documentclass[a4paper,11pt,twocolumn]{book}
\usepackage[T1]{fontenc}
\usepackage[utf8]{inputenc}
\usepackage{lmodern}
\usepackage{amsmath}
\usepackage{amsfonts}
\usepackage{amssymb}
%%\usepackage{amsthm}
\usepackage{graphicx}
\usepackage{xcolor}
\usepackage{url}
\usepackage{theorem}
\usepackage{textcomp}
\usepackage{listings}
\usepackage{hyperref}
\usepackage{parskip}

\title{Prolog, la reprise}
\author{Bernard Tatin}
\date{\today}

\newtheorem{defn}{Définition}
\newtheorem{thm}{Théorème}

\lstset{emph={%  
   {:-}%
     },emphstyle={\color{red}\bfseries\underbar}%
}%

\begin{document}

\maketitle
\tableofcontents

\chapter{les bases}
Commençons par quelque chose de simple:

\begin{lstlisting}[language=Prolog]
nat(X):-
	integer(X),
	X > -1.
\end{lstlisting}

A noter que la version suivante, plus \emph{naturelle}, provoque une erreur:

\begin{lstlisting}[language=Prolog]
nat(X):-
	integer(X),
	X >= 0.
\end{lstlisting}

Pour rappel, \texttt{,} est la \textit{conjonction} soit le \textit{et} logique, tandis que \texttt{;} indique \textit{disjonction}, soit le \textit{ou} logique.


\end{document}