\documentclass[a4paper,12pt]{book}

%% \usepackage[T1]{fontenc}
%% \usepackage{gillius2}
%% \usepackage[utf8]{inputenc}
%% \usepackage[francais]{babel}
%% \usepackage{marginnote}
%% \usepackage{makeidx}
%% \usepackage{graphicx}



\usepackage[T1]{fontenc}
%% \usepackage[utf8]{inputenc}
\usepackage[french]{babel}
\usepackage{lmodern}
\usepackage{CormorantGaramond}
\usepackage{amsmath}
\usepackage{amsfonts}
\usepackage{amssymb}
%%\usepackage{amsthm}
\usepackage{graphicx}
\usepackage{xcolor}
\usepackage{url}
\usepackage{theorem}
\usepackage{textcomp}
\usepackage{listings}
\usepackage{hyperref}
\usepackage{parskip}

\title{Prolog, la reprise}
\author{Bernard Tatin}
\date{\today}

\newtheorem{defn}{Définition}
\newtheorem{thm}{Théorème}

\usepackage{listings}
\usepackage{caption}
\usepackage{boxhandler}
\usepackage{color}

\definecolor{mygreen}{rgb}{0,0.6,0}
\definecolor{mygray}{rgb}{0.5,0.5,0.5}
\definecolor{mymauve}{rgb}{0.58,0,0.82}

\lstset{ %
  backgroundcolor=\color{white},   % choose the background color; you must add \usepackage{color} or \usepackage{xcolor}; should come as last argument
  basicstyle=\footnotesize\ttfamily,        % the size of the fonts that are used for the code
  breakatwhitespace=false,         % sets if automatic breaks should only happen at whitespace
  breaklines=true,                 % sets automatic line breaking
  captionpos=b,                    % sets the caption-position to bottom
  commentstyle=\color{mygreen},    % comment style
  %deletekeywords={...},            % if you want to delete keywords from the given language
  escapeinside={\%*}{*)},          % if you want to add LaTeX within your code
  extendedchars=true,              % lets you use non-ASCII characters; for 8-bits encodings only, does not work with UTF-8
  frame=single,	                   % adds a frame around the code
  keepspaces=true,                 % keeps spaces in text, useful for keeping indentation of code (possibly needs columns=flexible)
  language=Prolog,                 % the language of the code
  keywordstyle={\color{green}\bfseries\itshape},       % keyword style
  numbers=none,                    % where to put the line-numbers; possible values are (none, left, right)
  numbersep=5pt,                   % how far the line-numbers are from the code
  numberstyle=\tiny\color{mygray}, % the style that is used for the line-numbers
  rulecolor=\color{black},         % if not set, the frame-color may be changed on line-breaks within not-black text (e.g. comments (green here))
  showspaces=false,                % show spaces everywhere adding particular underscores; it overrides 'showstringspaces'
  showstringspaces=false,          % underline spaces within strings only
  showtabs=false,                  % show tabs within strings adding particular underscores
  stepnumber=2,                    % the step between two line-numbers. If it's 1, each line will be numbered
  stringstyle=\color{mymauve},     % string literal style
  tabsize=2,	                   % sets default tabsize to 2 spaces
  title=\lstname                   % show the filename of files included with \lstinputlisting; also try caption instead of title
}


\DeclareCaptionFont{white}{ \color{white} }
\DeclareCaptionFormat{listing}{
  \colorbox[cmyk]{0.43, 0.35, 0.35,0.01 }{
    \parbox{0.8\textwidth}{\hspace{15pt}#1#2#3}
  }
}
\captionsetup[lstlisting]{ format=listing, labelfont=white, textfont=white, singlelinecheck=false, margin=2em, font={footnotesize}}


\begin{document}

\maketitle
\tableofcontents

\chapter{les bases}

\section{le langage}
Commençons par quelque chose de simple:

\begin{lstlisting}[language=Prolog, caption=définition de l'ensemble des entiers]
nat(X):-
	integer(X),
	X >= 0.
\end{lstlisting}

Pour rappel, '\texttt{,}' est la \textit{conjonction} soit le \textit{et} logique, tandis que '\texttt{;}' indique \textit{disjonction}, soit le \textit{ou} logique.

\section{l'environnement \texttt{swipl}}

C'est simple à lancer: \texttt{swipl}. 

Pour profiter de répertoire contenant des bibliothèques personnelles, il faut, autant pour la compilation que pour l'interprétation, rajouter le paramètre:
\begin{lstlisting}[language=sh, caption=swipl: démarrer avec le chemin des bibliothèques] 
swipl -p library=$(pwd)/lib:...
\end{lstlisting}
 
à moduler en fonction de la configuration.

Pour créer un binaire exécutable, il faut lancer la commande:
\begin{lstlisting}[language=sh, caption=swipl: créer un binaire exécutable]
swipl -p library=... \\
    --goal=main \\
    --stand_alone=true \\
    -o binary \\
    -c sourcefiles ...
\end{lstlisting}

Pour des exemples plus complets, con\-sul\-tez le(s) \texttt{Makefile}(s) des ré\-per\-toi\-res sour\-ces.

\section{LogTalk}
\subsection{notes pour l'installation}
\subsubsection{\textit{Yap}}

L'installation se déroule très bien sous réserve de petites
modifications dans l'\textit{adapter} de \emph{yap}. 
La procédure \texttt{member} doit être préfixée
par \texttt{lists:} et la fonction \texttt{term\_hash}
doit être préfixée par \texttt{terms:}. 
Et là, tout devient normal.

\subsection{pourquoi \emph{LogTalk} ?}
Avec les \emph{catégories} et les objets, \emph{LogTalk} permet de mieux gérer les espaces de noms. Surtout, \emph{LogTalk} permet un portage du code entre plusieurs implémentations de \emph{Prolog} sans modification du code ou, au pire, à l'aide de directives de compilation conditionnelles. Voici un exemple de ces dernières:

\begin{lstlisting}[language=Prolog, caption=LogTalk compilation conditionnelle]
:- if(current_logtalk_flag(prolog_dialect, swi)).
    string_2_chars(Atom, Chars) :-
        string_chars(Atom, Chars).
:- else.
    string_2_chars(Atom, Chars) :-
        atom_chars(Atom, Chars).
:- endif.
\end{lstlisting}

\end{document}
